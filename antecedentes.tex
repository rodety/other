\documentclass[10pt,a4paper]{article}
\usepackage[spanish]{babel}
\usepackage{anysize}
\usepackage[latin1]{inputenc}
\usepackage{graphicx}
\usepackage[light,math]{iwona}
\usepackage[T1]{fontenc}
\marginsize{1cm}{0cm}{0cm}{0cm} 
\pagestyle{empty} 
\begin{document} 
\begin{minipage}{10cm}
\end{minipage}
\hfill
\begin{minipage}{10cm}
\begin{flushleft}
\scalebox{1.4}[1.5]{
\begin{tabular}{|p{3cm}p{2.3cm}|}
\hline 
\multicolumn{2}{|l|}{\scriptsize{\textbf{PACIENTE}}}\\
\multicolumn{2}{|p{5.3cm}|}{\scriptsize{Apellidos: Danko Antolovic}}\\ 
\multicolumn{2}{|p{5.3cm}|}{\scriptsize{Nombres: Juan Alberto}}\\ 
\multicolumn{2}{|p{5.3cm}|}{\scriptsize{M\'edico Tratante: Carlos arriaga Montesinos}}\\ 
\hline 
\multicolumn{2}{|p{5.3cm}|}{\scriptsize{Servicio: Cirujia Estetica}}\\
\hline 
\scriptsize{Historia Cl\'inica: 00-458-632}&\scriptsize{Cama: 0012} \\ 
\hline 
\end{tabular}}
\end{flushleft}
\end{minipage}

\vspace{0.5cm}
\scalebox{1.40}[1.55]{
\begin{tabular}{|p{2cm}p{2cm}p{3cm}p{2cm}|} 
\hline 
\multicolumn{1}{|l}{\scriptsize{EDAD: 19}}& \multicolumn{2}{l}{ \scriptsize{FECHA DE NACIMIENTO: 12/02/1999}}& \multicolumn{1}{l|}{ \scriptsize{FECHA DE INGRESO: 03/12/12}}\\
\hline 
\multicolumn{1}{|l}{\scriptsize{SEXO: masculino}}& \multicolumn{1}{c}{ \scriptsize{ESTADO CIVIL: casado}}& \multicolumn{2}{p{7cm}|}{ \scriptsize{OCUPACI\'ON: Licenciado en Ciencias de la Computacion y administraodor de finanzas}}\\  
\hline 
\multicolumn{4}{|l|}{\scriptsize{ANTECEDENTES PATOL\'OGICOS PERSONALES:}}\\
\multicolumn{4}{|p{12.5cm}|}{\scriptsize{FFTW is a library of C functions which compute discrete Fourier transforms. The library is fairly comprehensive: it computes complex and real Fourier transforms in any number of dimensions, and it has single-precision and double-precision forms. FFTW was first developed for Unix, but a Windows implementation is available as well. It is distributed as free software, under GNU General Public License.}}\\ 
 & && \\
\hline 
\multicolumn{4}{|l|}{\scriptsize{ALERGIAS:}}\\ 
\multicolumn{4}{|p{12.5cm}|}{\scriptsize{FFTW utilizes the so-called Fast Fourier algorithm, which can improve the speed of the computation, dependent upon the number of input data points. This document demonstrates how to use FFTW in your program, and illustrates the concepts behind FFTW, by guiding you through a typical example. A comprehensive manual and installation instructions can be found at the FFTW web site:}}\\ 
\hline 
\multicolumn{4}{|l|}{\scriptsize{FECHA DE \'ULTIMA MENSTRUACI\'ON:--}}\\ 
\hline 
\multicolumn{4}{|l|}{\scriptsize{R\'EGIMEN CATAMENIAL:--}}\\
\hline 
\multicolumn{1}{|l}{\scriptsize{H\'ABITOS NOCIVOS:}}& \multicolumn{1}{l}{\scriptsize{ALCOHOL(X)}}&\multicolumn{2}{l|}{\scriptsize{TABACO( )}}\\
\hline 
\multicolumn{4}{|l|}{\scriptsize{MEDICACI\'ON ACTUAL:}}\\
\multicolumn{4}{|p{12.5cm}|}{\scriptsize{One property of the Fourier transformation that is very useful in graphics and imaging is known as the convolution theorem. Technically, a convolution of two functions is an integral of their product, where one function is displaced relative to the other. }}\\ 
\hline 
\multicolumn{4}{|l|}{\scriptsize{OPERACION (FECHA, HOSPITAL): }}\\
\multicolumn{4}{|p{12.5cm}|}{\scriptsize{At the end of the calculation, we free up the memory that was taken up by the Fourier transform:}}\\ 
\hline 
\multicolumn{4}{|l|}{\scriptsize{ANTECEDENTES PATOL\'OGICOS FAMILIARES:}}\\
\multicolumn{4}{|p{12.5cm}|}{\scriptsize{There are many good descriptions of the Fourier transformation in the literature. Here are some references, to help you become familiar with the topic:}}\\ 
\hline 
\multicolumn{4}{|l|}{\scriptsize{OBSERVACIONES:}}\\
\multicolumn{4}{|p{12.5cm}|}{\scriptsize{Once you have FFTW installed on your system, link to the library file (fftw3,  or whatever variation applies to your particular installation), and include the usual header file in your code:}}\\ 
\hline 
\end{tabular}}
\vspace{0.6cm}
\hspace{13cm}
\begin{flushleft}
\begin{LARGE}
\textbf{ANTECEDENTES}
\end{LARGE}
\end{flushleft}
\end{document}
